\documentclass[12]{article}
\usepackage[utf8]{inputenc}
\usepackage[a4paper,margin = 1in]{geometry}
\usepackage{enumitem}
\usepackage{color}
\usepackage{amssymb,amsmath}

\newcommand{\re}{\textbf{Response: }}
\newcommand\bd[1]{{\color{blue}#1}}
\newcommand{\bbeta}{\boldsymbol{\beta}}
\newcommand{\boldeta}{\boldsymbol\eta}

\begin{document}

\section*{Response to Reviewers}
\subsection*{Manuscript ID SMM-22-0007: Cross Validation Approaches for
Penalized Cox Regression}

The authors wish to thank the reviewers for their careful reviews and thoughtful comments.

\subsection*{Response to Reviewer 1's Comments}

The authors compare four different cross-validation options for the important problem of penalized Cox regression, including two new proposed methods. The methodology is nicely described, and the illustration in Figure 1 should prove helpful to readers new to this area. The interpretations of the simulation results are helpful, and a good choice for the application dataset was made. The take-home message of the usefulness of the linear predictor approach was clearly made based on the work performed. However, there are a number of important questions and comments to mention here that should be addressed:

\begin{enumerate}[align = left]
\item On p.4, roughly line 34, you have the statement “although the methods we analyze can be used with any penalty”.  This statement should be further elaborated, providing some tangible examples (including beyond the straightforward elastic net approach considered in the application analysis).

\re Thank you for this comment. We provided a few examples of penalty functions, such as the elastic net penalty, smoothly clipped absolute deviations (SCAD) penalty, and minimax concave penalty (MCP) on page 4. We also included a paragraph for more general application in the Discussion section.

\item Though you aim to make various considerations in your simulation study, the impact of this work may very well be limited given consideration of only exponential survival distribution and independent censoring. 
    \begin{enumerate}
        \item Extensions into time-varying covariates, and, preferably, joint modeling, would be useful as well. At least the first two items (more flexible survival distributions, and dependent censoring) should be explored in your work, while perhaps extensions into joint modeling to accommodate one or more longitudinal predictors could be discussed as future work in your Discussion.
        
        \re Thank you for making this comment. This work aims to make comparisons among cross-validation methods, which are more commonly used in the model selection context. In the Discussion section, we included a paragraph to state that the methods in this manuscript can be applied to any model selection process that involves the cox partial likelihood. While cross-validation and model selection play a key role for fitting penalized cox regression models, the process of fitting time-varying covariates and joint modeling with longitudinal covariates do not always involve model selection, therefore this extension is out of the main focus of this work. 
        
        \item Regarding distributional form, the exponential form here is quite limiting, and even common choices for a more complex distributional form, such as Weibull and Gompertz, might be considered too limiting. See, for example, Harden and Kropko (2019).
        
        \re Thank you for this comment. We included additional simulation results in the supplementary material where the baseline hazard is generated from Weibull and Gompertz distribution. The overall conclusions do not change for both of these more complex distributional form.
        
    \end{enumerate}

\item Speaking of your Discussion section, there is no mention of any limitations of the current work. Aside from the joint modeling comment above, you should also talk about violations of linearity, and of non-proportional hazards. That is, you need to provide a number of additional sentences, and references, as warranted, to talk about how the current contributions could ultimately be made more generalizable to various survival dataset characteristics, and not just those with a very simplifying structure. Some discussion on how to think about survival datasets with rare events should be commented on as well.

\re Thank you for the comments. The violation of linearity is usually handled via selecting or fitting a non-linear functional form for the covariates. The approaches that are discussed here are built upon Cox partial likelihood, which do not deal with non-proportional hazards and rare events, therefore those topics are out of the scope of this manuscript.  We included sentences in the Discussion section to address those points.

\item On p.10, second sentence first paragraph of Section 3.1, should mention that the specifics of the different conditions considered (in the simulation study) are described below.

\re Thank you for the suggestion. We added the sentence to the first paragraph of Section 3.1.

\item In Section 3.1, p.11, first line (and in subsequent uses below), I would suggest adding a $t_0$ index to $y$, since the definition of $y$ is a direct function of $t_0$.

\re Thank you for the suggestion. We added $t_0$ index to $y$ notation.

\item And only looking at one $t_0$ in your simulation seems limiting here (for the Brier and KL score calculations). Though if just one choice is used it makes sense to use the median, explain why only one choice of $t_0$ is considered.  This would be especially true if you consider other more complicated survival distributions.

\re Thank you for this comment. We provided additional simulation results in the supplement with more choices of $t_0$ with Exponential, Weibull and Gompertz distributions. The overall conclusions do not change with various $t_0$.

\item Was confused on a point made in the second to final sentence on p.12. It says “In all settings, the censoring rate was 30$\%$”. Is this just set at 30$\%$ for the additional simulations, noting that you just described a setting in the prior paragraph with 10$\%$ expected censoring. Please clarify, including in the text. 

\re Thank you for catching this. We clarified in the writing that this censoring rate only applies to settings in Table 1.

\item There is an interesting semantic point to raise. You comment on the approaches taking/accepting fewer variables as conservative and, thus, those accepting more variables as liberal. However, an argument can be made that a method making fewer decisions, i.e., not throwing out as many variables as another, as being more cautious/conservative. It’s worth elaborating on this point.

\re \bd{To do: address this in Discussion}

\item In the application section, you mention fitting elastic net models to the gene expression dataset, since elastic net does a better job than traditional Lasso when modeling many correlated genes. However, there is evidence that newer Lasso approaches (e.g., Precision Lasso: Wang et al, 2019, Bioinformatics) outperform elastic net models in this setting, so you need to justify why only the elastic net is considered as the only alternative to traditional Lasso here. Perhaps you can mention potential alternatives to elastic net as an improvement in Section 4, and include this as an area of future focus in the Discussion.

\re Thank you for the suggeston. We included the precision lasso method in the discussion section.

\item Regarding elastic net, it was not justified why other alpha values (beyond 0.3, 0.5, 0.7, and 0.9) are not considered in the analysis, such as a denser grid between 0.1 and 0.5, particularly after seeing the results (in Table 3) from the sparse selection of alpha values used. More alpha values should be considered.

\re \bd{maybe the easiest way is to use more alpha values in this example?}

\item There are several typos sprinkled throughout the paper, so a careful proofreading is required. Some examples include: (i) Section 3.1, second paragraph, where the phrase “the value of $\beta$ used the generate the data” requires editing. (ii) Table 1, Scenario 4, would seem to be for
strong signal, not weak. (iii) p.22, “more clear” should be changed to “clearer”.

\re Thank you for catching those. We made edits in the manuscript.

\end{enumerate}

\subsection*{Response to Reviewer 2's Comments}
In this manuscript, the authors proposed two new cross-validation methods for Cox proportional hazards regression.

\begin{enumerate}[align = left]
\item Throughout the paper, the authors mainly (almost only) advocate for use of one of the proposed methods, “cross validated linear predictors”. Also based on numerical studies, I see no point in presenting the other proposed method, “cross validated deviance residual”. If the authors want to present both proposed methods, the cross validated deviance residual method should be better highlighted in numerical studies and Discussion for its advantages, which is not sufficiently illustrated in the current version of paper.

\re \bd{maybe one advantage is that the deviance residuals can be applied to a more wide range of survival models other than the regression approaches. say, for a random forest model with survival outcome, they can't really use the linear predictor approach right?}

\item Presentation of simulation studies is quite confusing so should be streamlined. Specifically, data characteristics such as sample size and censoring rate are not consistent across simulation scenarios considered for each of different evaluation metrics in Section 3.

    \begin{enumerate}
    \item I suggest presenting all the simulation scenarios varying different data characteristics at the beginning of Section 3 and discuss the comparative performance of the methods using each of model evaluation metrics in the following subsections. Addressing this comment will require additional computations for Sections 3.3-3.5
    
    \re \bd{I think there are definitely values to present a different variety of simulation scenarios. but maybe we should re-structure the writing a bit more}

    \item Considering the censoring rate of 50$\%$ in the application, I would consider scenarios with a higher censoring rate (say 50$\%$) than what was considered in Sections 3.1 and 3.2 (10$\%$ and 30$\%$). This will also help justify authors’ argument in P17L34. 
    
    \re We explored the impact of censoring in Section 3.1, varying from 40$\%$ and to 80$\%$. The censoring proportion has the greatest impact on the stability for the basic method, but does not impact the overall conclusions for the other methods.

    \item Table 2 - Why only present the number of true positives here? I suggest presenting other operating characteristics such as TPR, NPR, PPV, etc... Then the results might help better justify the findings from the application. For example, depending on the results, referring to the simulation results could strengthen the argument given at the end of the second paragraph in Discussion (which is quite speculative in my opinion).
    
    \re \bd{one argument would be we are evaluting the predictive performance. This comment can be addressed jointly with Reviewer 1's comment 8}
    
\end{enumerate}

\item Did the authors make the software available for the proposed approaches?

\re The cross-validated linear predictor approach is implemented in the \texttt{ncvreg} package. The basic approach and the Verweij and Van Houwelingen's approach are available with the \texttt{glmnet} package. \bd{are we submitting the R codes as supplementary material?}

\item Many places require major/minor clarification and corrections:
\bd{the following points are all edited in the manuscript}
\begin{itemize}[align = left]
\item[-]  P3L33: define beta

\re Thank you for the suggestion. We included definition for $\bbeta$ in this paragraph when first introducing the notation.

\item[-] Distinguish between scalars and vectors by using bold letters for vectors

\re Thank you for the suggestion. We updated the manuscript by using bold betters for vectors.

\item[-] Figure 1: Since “Folder 1” is highlighted for each panel, I think the figure corresponds to the specific case with k=1. If so, k should be replaced by 1 in expressions.

\re Thank you for this careful observation. In the figure, we used "Folder 1" for the simplicity of drawing. However, replacing $k$ with $1$ in the expressions in the figure may cause confusion to the readers. Therefore, we added description in the caption of Figure 1 to clarify that this illustration represents $k = 1$.

\item[-]  P8L5: I would not use “k” in the expression because k is used for k-th “fold” in this paper.

\re Thank you for the suggestion. We replaced $k$ with $\tau$ in that formula

\item[-]  Equation (6) and P8L10: I found the definition of $\hat{\eta}^{\text{cv}}$ confusing. Do we not need some step for pooling cross-validated linear predictors over all K folds?

\re No additional steps are needed to pool the cross-validated linear predictors over all K folds. Please refer to the linear predictor diagram in Figure 1. Suppose each fold has $n_k$ number of observations and the complete data have $n = \sum_{k=1}^{K}  n_k$ observations. The cross-validated linear predictor of the $k$th folder $\hat{\boldeta}^{\text{cv}}_k$ is a vector of length $n_k$ and the complete set of cross-validated linear predictors $\hat{\boldeta}^{\text{cv}}$ has a length of $n$. We also updated the vector $\boldeta$ by using bold.
 
\item[-]  Table 1: what are the numbers in parentheses? 

\re The numbers in the parentheses are standard deviation across all simulated samples, which is a metric for simulation errors.

\item[-] Figure 4 caption: LOO is not defined

\re Thank you for the suggestion. We updated the caption to include defition for LOO.

\item[-]  Figure 4 caption: (blue line) should be (green line)?

\re Thank you for catching this. We corrected this in the caption.

\item[-]  Figure 4 legend: The “Method” can mean two different things in this paper – either (four different CV methods) or (10 fold CV, LOOCV). Therefore, the legend is very confusing because “Basic” here actually means basic CV with 10 folds. Although the text in the paper clarifies this, tables and figures need to be understandable by themselves without reference to the text.

\re Thank you for the suggestion. We edited the caption of Figure 4 to clarify the legends.

\item[-]  Table 4, what are “Selected” and “P” here?

\re The column "Selected" represents the number of genes that are selected by the linear predictor cross-validation approach; each row represents the numbder of genes that fall into that function category. $p$ is the p-value. \bd{is the p-value adjusted? update $p$ with "p-value"?}

\end{itemize}
\end{enumerate}



\subsection*{References}

Harden JJ, Kropko J (2019). Simulating duration data for the Cox model, Political Science Research and Methods, 7(4): 921-928. https://doi.org/10.1017/psrm.2018.19

Wang H, Lengerich BJ, Aragam B, Xing EP (2019). Precision Lasso: accounting for correlations and linear dependencies in high-dimensional genomic data. Bioinformatics, 35(7), 1181-1187. https://doi.org/10.1093/bioinformatics/bty750

\end{document}
